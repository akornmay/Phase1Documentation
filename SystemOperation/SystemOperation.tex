
\ifx\isEmbedded\undefined

%%%%% copy from main
\documentclass[twoside,openright,titlepage,a4paper,11pt,chapterprefix,appendixprefix]{scrreprt}%
%\usepackage[ngerman]{babel} % deutsche Silbentrennung
\usepackage[ansinew]{inputenc} % wegen deutschen Umlauten
\usepackage{graphicx}
\usepackage{subfigure}
\subfigcapmargin = 0.2cm
\usepackage{mathcomp}
\usepackage{amsmath}
\usepackage[format=plain ,margin={1cm,1cm}]{caption} %koma script ist standartm��ig auf "hang", links und rechts einger�ckt
\usepackage{chapterfolder}
% and we re-write includegraphics
\let\includegraphicsWithoutCF\includegraphics
\renewcommand{\includegraphics}[2][]{\includegraphicsWithoutCF[#1]{\cfcurrentfolder#2}}


\pagestyle{headings} % wir wollen auf jeder Seite eine Ueberschrift

\setlength{\unitlength}{1cm}
\setlength{\oddsidemargin}{0.3cm}
\setlength{\evensidemargin}{0.3cm}
\setlength{\textwidth}{15.5cm}
\setlength{\topmargin}{-0.7cm}
%\setlength{\textheight}{22cm} %bei seitlich
\setlength{\textheight}{23cm} %bei mittig
%%%%%%%%%%%%%%%%%%%%%%%%%%%%%%%%%%%%%%%%%%%%%%


\begin{document}


\else
\fi
% ******************************************************************************

This section will show how one can run and operate the system to do basic tests.
For now it only consists of the basics tools we developed at CERN but as soon as possible POS should also be integrated.

The text follows the instructions found on Mia Liu's twiki \url{https://twiki.cern.ch/twiki/bin/viewauth/CMS/PixelPhase1SetupMicroTCA}.

All examples are made for the TIF test stand for now but should be kept as generic as possible.

\section{Setting up the CCU and Portcard}

First make sure the correct environment variables have been picked up.
 On \texttt{cmsuppixpc001} you would do for example
\code{source ~/setup.sh}

We are working in the following with the BPixelTools version that can run on the microTCA system.
 Change the directory there and also source the needed environment variables.
\code{cd  ~/FEC\_SW/Pixel/BPixelTools/}
\code{source setenv.sh}

Then change to the CCU directory
\code{cd ccu}
\\From here we configure the CCU by running
\code{bin/ccu -utca ~/FEC\_SW/connections.xml -fechardwareid board -ccu 0x3f -channel 0x11}
\\This will open an interactive shell where first
\code{scanccu}
\\and
\code{scanringdevice}
\\needs to be run.
If everything worked fine and there were no errors the Portcard can be initialised by running the script
\code{Init4ChPortcard\_uTCA.sh}

To control the sending of triggers with the TTCci system we have to start HyperDAQ.
 To do this, change to the directory
\code{cd /home/fectest/FEC\_mTCA/pixel/PixelRun}
\\There run the appropriate shell script
\code{./run\_noRU\_TIF.sh}
\\From the browser \url{IStillNeedTheURL} you can control the TTCci sytstem then.


\section{Running the microTCA FED}






% ******************************************************************************
\ifx\isEmbedded\undefined
\input{../biblio.tex}
\end{document}
\else
\fi
